%-----------------------------------------------------------------------------%
%- Logic :: Propositional Logic and Truth Tables
%-----------------------------------------------------------------------------%
\section{Propositional Logic and Truth Tables}
\label{sec:PropositionalLogicAndTruthTables}

A proposition is a statement that is either true or false.

``It is Thursday'' is a statement, which may be true or false depending on the
day.
``Pigs can fly'' is a statement, which is most likely false unless something odd
is done to the pig.
$2 + 3 = 5$ is a statement which turns out to be true (T).
$2 + 3 = 7$ is a statement which turns out to be false (F).
$ -3 < 7  $ (T)
$\frac{5}{4} > \frac{6}{5}$ (T)

Some things which are not propositions:
\begin{itemize}
  \item ``Go away.'' is a statement, but cannot be resolved to a true or false
  value. ``Because it's Monday''. This is only a sentence fragment 
  \item $ x + y $
  \item ``I am lying.'' is a paradox. If you actually are lying, then the
  statement is true and thus no longer a lie (and hence the truth). Paradoxes come about due to
the notion of ``self-reference''.
  \item ``This sentence is false'' is another example. Interestingly though,
  ``This sentence is true'' is not a paradox.
  \item ``Nothing is true'' is another paradox \footnote{and is an oft used
  topic in post-modernism, but that's arts topic and best left to people who
  belong in Y3A}.
  \item Russell's Paradox
\end{itemize}

Most things that we want to say contain more than a single idea so we must
combine, so we must combine propositions in some way.

Propositions are often represented by the letters p and q. They can be combined
using things called ``connectives'' like:
\begin{itemize}
  \item and
  \item or
  \item not
  \item xor
  \item xnor
  \item implies
  \item iff (if and only if)
  \item equivalent
\end{itemize}

Let ``p'' be the statement ``I like tea'' and ``q'' be the statement ``I like
coffee''.

If I wanted to say: ``I like tea and coffee'': $p \land q$
If I wanted to say: ``I like tea or coffee'': $p \lor q$
If I wanted to say: ``I like tea but not coffee'': $p \land \lnot q$
``I don't like tea or coffee'' can be written two ways: $ -(p \lor q)$ or $\lnot p
\land \lnot q $

Let ``p'' be the statement ``It is raining.''
Let ``q'' be the statement ``I need an umbrella.''

If it is raining, then I need an umbrella:

$ p \to q $

ie, the fact that it is raining implies that I need an umbrella.

If it is not raining, so I don't need an umbrella.

$ \lnot p \to \lnot q $

What about $q \to p$?
This says: ``If I need an umbrella, then it is raining.'' This is not
necessarily true as there may be many uses for an umbrella besides keeping rain
off\footnote{eg an umbrella salesman or as a parachute}.

Let ``p'' be the statement ``he is vegetarian''
Let ``q'' be the statement ``he eats no meat''

He is vegetarian if and only if he eats no meat.

$ p \iff q \equiv (p -> q) \land (q -> p)$
This establishes equivalence between proposition p and q.


Conjunction $p \land q$ is true only when both p,q are true.
\begin{table}[!htb]
\label{tab:TruthTableAND}
\begin{tabularx}{\linewidth}{| c | c | X |} \hline
  p & q & $(p \land q)$ \\ \hline \hline
  F & F & F \\ \hline
  F & T & F \\ \hline
  T & F & F \\ \hline
  T & T & T \\ \hline
\end{tabularx}
\caption{Truth Table: logical AND}
\end{table}

Disjunction $p \lor q$ is true when at least one of p,q are true:
\begin{table}[!htb]
\label{tab:TruthTableOR}
\begin{tabularx}{\linewidth}{| c | c | X |} \hline
  p & q & $(p \lor q)$ \\ \hline \hline
  F & F & F \\ \hline
  F & T & T \\ \hline
  T & F & T \\ \hline
  T & T & T \\ \hline
\end{tabularx}
\caption{Truth Table: logical OR}
\end{table}

Negation $\lnot p$ is true when $p$ is false
\begin{table}[!htb]
\label{tab:TruthTableNOT}
\begin{tabularx}{\linewidth}{| c | X |} \hline
  p & $(\lnot p)$ \\ \hline \hline
  F & T \\ \hline
  T & F\\ \hline
\end{tabularx}
\caption{Truth Table: logical NOT}
\end{table}

Conditional $p \to q$ if p then q, or ``p implies q''. This is false when
$p$ is true and $q$ is false with a default condition of true.

\begin{table}[!htb]
\label{tab:TruthTableIMPLIES}
\begin{tabularx}{\linewidth}{| c | c | X |} \hline
  p & q & $(p \to q)$ \\ \hline \hline
  F & F & T \\ \hline
  F & T & T \\ \hline
  T & F & F \\ \hline
  T & T & T \\ \hline
\end{tabularx}
\caption{Truth Table: logical IMPLIES}
\end{table}

A plain English example: It can't rain when there are no clouds, so ``if it is
raining then it is not cloudy'' is false.

Suppose it isn't raining, it could still be cloudy. This remains true.

Converse $q \to p$
\begin{table}[!htb]
\label{tab:TruthTableCONVERSE}
\begin{tabularx}{\linewidth}{| c | c | X |} \hline
  p & q & $(q \to p)$ \\ \hline \hline
  F & F & - \\ \hline
  F & T & - \\ \hline
  T & F & - \\ \hline
  T & T & - \\ \hline
\end{tabularx}
\caption{Truth Table: logical CONVERSE}
\end{table}

If it is cloudy then it is raining (not always true).
q->p is NOT the same as $p -> q$

Contrapositive $\lnot p -> \lnot q$
Use to establish the notion of ``proof by contradiction''.

Double conditional $ p \equiv q $
p if and only if q: p iff q
$p \to q$ and $q \to p$ must both be true
That is $(p \equiv q) \equiv (( p \to q) \land (q \to p))$

\begin{table}[!htb]
\label{tab:TruthTableEQUIVALENCE}
\begin{tabularx}{\linewidth}{| c | c | X |} \hline
  p & q & $( p \equiv q ) \equiv (( p \to q) \land (q \to p))$
                                                    \\ \hline \hline
  F & F & F \\ \hline
  F & T & F \\ \hline
  T & F & F \\ \hline
  T & T & T \\ \hline
\end{tabularx}
\caption{Truth Table: logical DOUBLE CONDITIONAL}
\end{table}

eg let p: 4 is an even number
       q: 4 is divisible by 2
       p and q clearly mean the same thing.
       
Exclusive or (xor) $p \lxor q$
This is true when either p,q is true, but not both.

\begin{table}[!htb]
\label{tab:TruthTableXOR}
\begin{tabularx}{\linewidth}{| c | c | X |} \hline
  p & q & $( p \lxor q )$ \\ \hline \hline
  F & F & F \\ \hline
  F & T & T \\ \hline
  T & F & T \\ \hline
  T & T & F \\ \hline
\end{tabularx}
\caption{Truth Table: logical XOR}
\end{table}

How large should you make your truth table?
\begin{enumerate}
  \item For every variable we have 2 possible values: True and False.
  \item If we have n propositions we need ${2}^{n}$ rows in the truth table.
\end{enumerate}
For example, if we have the variables $p,q,r$ then we need ${2}^{3} = 8$ rows.

% TODO: fix order of operations

Order of operations:

\begin{enumerate}
  \item Do OR connectives
  \item Do AND connectives
\end{enumerate}


Two expressions are logically equivalent if their truth values match for every
combination of the truth values of their atomic propositions\footnote{an atomic
proposition is a proposition cannot be broken down any further}. This is the
same as saying the expressions have the same truth tables.

We want to show $p \land (q \lor r) \equiv ( (p \land q) \lor (p \land r) )$:
\begin{enumerate}
  \item Build truth table. We need 8 rows.
\end{enumerate}

Tautology: In ordinary language it is when you use redundant words which are
unnecessary \footnote{This explanation in itself is a tautology where redundant
and unnecessary mean the same thing}. A tautology is always true. The opposite
of a tautology is a contradiction; which is always false.

Tautology example: $p <-> q $ iff 

\begin{table}[!htb]
\label{tab:TruthTableTautology}
\begin{tabularx}{\linewidth}{| c | c | c | c | X |} \hline
  $p$ & $q$ & $(p \land q) $ & $ \equiv $ & $ (\lnot p \lor \lnot q) $
                                                      \\ \hline \hline
   F  &  F  &               &       &                 \\ \hline
   F  &  F  &               &       &                 \\ \hline
   F  &  F  &               &       &                 \\ \hline
   F  &  F  &               &       &                 \\ \hline
\end{tabularx}
\caption{Truth Table: Tautology}
\end{table}