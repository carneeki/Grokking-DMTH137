%-----------------------------------------------------------------------------%
%- Laws of Logic
%-----------------------------------------------------------------------------%
\chapter{Laws of Logic}
\label{sec:LawsOfLogic}

\begin{align}
  p \land q & 
    & \equiv q \land p
  \label{eq:Commut}
\end{align}

\begin{align}
  p \land (q \land r) &
    & \equiv (p \land q) \land r
  \label{eq:Assoc}
\end{align}

\begin{align}
  p \land p &
    & \equiv p
  \label{eq:Idemp}
\end{align}

\begin{align}
  p \lor p &
    & \equiv p
  \label{eq:Idemp2}
\end{align}

\begin{align}
  p \land (p \lor q) &
    & \equiv p
  \label{eq:Distrib}
\end{align}

\begin{align}
  p \lor (p \land q) & \equiv p
  \label{eq:Distrib2}
\end{align}

\begin{align}
  p \land \mathbb{T} & \equiv p
  \label{eq:IdentAND}
\end{align}

\begin{align}
  p \lor \mathbb{F} & \equiv p
  \label{eq:IdentOR}
\end{align}

\begin{align}
  p \land \mathbb{F} & \equiv F
  \label{eq:Annihil}
\end{align}

\begin{align}
  p \lor \mathbb{T} & \equiv T
  \label{eq:Annihil2}
\end{align}

\begin{align}
  p \land \not p & \equiv \mathbb{F}
  \label{eq:Inv1}
\end{align}

\begin{align}
  p \lor \not p & \equiv \mathbb{T}
  \label{eq:Inv2}
\end{align}

\begin{align}
  -(p \land q) &
    & \equiv -p \lor -q
  \label{eq:deMorgan}
\end{align}

\begin{align}
  -(p \lor q) &
    & \equiv -p \land -q
  \label{eq:deMorgan2}
\end{align}

\begin{align}
  p \to q &
    & \equiv -p \lor q
  \label{eq:Impl}
\end{align}

\begin{align}
  -(-p) &
    & \equiv p
  \label{eq:2Neg}
\end{align}

\begin{align}
  p \leftrightarrow q
    & \equiv (p \to q) \land (q \to p)
  \label{eq:Equiv}
\end{align}
A more complete table of laws can be found on Wikipedia 
%\footnote{
%url{http://en.wikipedia.org/wiki/Propositional_logic#Basic_and_derived_argument_forms}}


% TODO: provide some examples:
Examples:
$ p \land (p \lor q) \equiv p$
  % TODO: this table

Distributive:
$ p \land (q \lor r) \equiv (p \land q) \lor (p \land r)$
  % TODO: this table

De Morgan:
$ -(p \land q) \equiv (-p \lor -q)$
  % TODO: this table
  
Brackets, Negation, OR, AND

Implication:
$ (p \to q) \equiv (-p \lor q)$
\begin{table}[!htb]
\label{tab:TruthTableTautology}
\begin{tabularx}{\linewidth}{| c | c | c | c | c |} \hline
  $p$ & $q$ & $(p \to q)$ & $\equiv$ & $(-p \lor q)$ \\
  0   &   0 &           T &          & \\
  0   &   1 &           F &          & \\
  1   &   0 &           F &          & \\
  1   &   1 &           T &          & \\
\end{tabularx}
\end{table}

We can use the laws of logic to simplify expressions (eg to process a
conditional more efficiently). \\

Example: simplify $(p \to q) \to r$ \\
We know that:
\begin{align}
  (p \to q) \to r
    & \equiv (-p \lor q) \to r       && \neqref{eq:Impl}\\
    & \equiv -(-p \lor q) \lor r     && \neqref{eq:Impl}\\
    & \equiv (-(-p) \land -q) \lor r && \neqref{eq:deMorgan}\\
    & \equiv (p \land -q) \lor r     && \neqref{eq:deMorgan}\\
\end{align}
(recall that L10 = implication)

Simplify $(p \to q) \to p$:
\begin{align}
 (p \to q) \to p
   & \equiv (-p \lor q) \to p       && \neqref{eq:Impl}\\
   & \equiv -(-p \lor q) \lor p     && \neqref{eq:Impl}\\
   & \equiv (-(-p) \land -q) \lor p && \neqref{eq:deMorgan}\\
   & \equiv (p \land -q) \lor p     && \neqref{eq:2Neg}\\
   & \equiv p L4                    && \neqref{eq:IdentAnd}
\end{align}

Simplify $(p \lor -(-p \to q))$:
\begin{align}
  (p \lor -(-p \to q)) & \\
       & \equiv p \lor -(-(-p) \lor q)         && \neqref{eq:Impl}\\
       & \equiv p \lor -(p \lor q)             && \neqref{eq:2Neg}\\
       & \equiv p \lor (-p \land -q)           && \neqref{eq:deMorgan}\\
       & \equiv (p \lor -p) \land (p \lor -q)  && \neqref{eq:Distrib}\\
       & \equiv (\mathbb{T}) \land (p \lor -q) && \neqref{eq:Inverse}\\
       & \equiv (p \lor -q)                     && \neqref{eq:Identity}
\end{align}

By simplifying $p \to (q \to (p \land q)) $:
%\begin{align}
%  p \to (q \to (p \land q))
%    & \equiv -p \lor (q \to (p \land q))      && \neqref{eq:Impl}\\
%    & \equiv -p \lor (-q \lor (p \land q))    && \neqref{eq:Impl}\\
%  \breaktext{something}
%    & \equiv (-p \lor -q) \lor (p \land q)    && \neqref{eq:Assoc}\\
%    & \equiv -(p \land q) \lor (p \land q)    && \neqref{eq:deMorgan}\\
%    & \equiv \mathbb{T}\\
%    & \therefore Tautology
%\end{align}

Show that $(p \lor q) \to r \equiv -r \to -(p \lor q)$:
\begin{align}
  (p \lor q) \to r & \equiv -r \to -(p \lor q) \\
  \intertext{LHS:}
  (p \lor q) \to r & \equiv - (p \lor q) \lor r      && \neqref{eq:Impl}\\
  \intertext{RHS:}
  -r \to -(p \lor q) & \equiv -(-r) \lor -(p \lor q) && \neqref{eq:Impl}\\
    & \equiv r \lor -(p \lor q)                      && \neqref{eq:2Neg}\\
    & \equiv -(p \lor q) \lor r                      && \neqref{eq:Assoc}\\
    & \equiv LHS  
\end{align}

Simplify $ -(p \to (p \lor q)) $:
\begin{align}
  -(p \to (p \lor q)) & \\
  & \equiv -(-p \lor (p \lor q))   && \neqref{eq:Impl}\\
  & \equiv -((-p \lor p) \lor q)   && \neqref{eq:Assoc}\\
  & \equiv -(\mathbb{T} \lor q)    && \neqref{eq:Inverse}\\
  & \equiv -\mathbb{T} && \\
  & \therefore \equiv \mathbb{F} && \\
  & \therefore contradiction
\end{align}

Establish the contrapositive: $ (p \to q ) \equiv (-q \to -p)$
\begin{align}
  \intertext{LHS:}
  p \to q & \equiv -p \lor q         && \neqref{eq:Impl}\\
  \intertext{RHS:}
  -q \to -p & \equiv -(-q) \lor (-p) && \neqref{eq:Impl}\\
            & \equiv q \lor -p       && \neqref{eq:2Neg}\\
            & \equiv -p \lor q       && \neqref{eq:Assoc}
\end{align}

Simplify $p \lor -(p \to -(q \land p))$:
\begin{align}
  p \lor -(p \to -(q \land p)) & \equiv p \lor -(p \to (-q \lor -p))
                                          && \neqref{eq:deMorgan}\\
  & \equiv p \lor -(-p \lor (-q \lor -p)) && \neqref{eq:Impl}\\
  & \equiv p \lor -(-p \lor -q \lor -p )  && \neqref{eq:Assoc}\\
  & \equiv p \lor -(-p \lor -p \lor -q)   && \neqref{eq:Commut}\\
  & \equiv p \lor -(-p \lor -q)           && \neqref{eq:Idempotent}\\
  & \equiv (-(-p) \land -(-q))            && \neqref{eq:deMorgan}\\
  & \equiv p \lor (p \land q)             && \neqref{eq:2Neg}\\
  \therefore & \equiv p                   && \neqref{eq:Absorb}
\end{align}