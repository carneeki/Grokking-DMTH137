%-----------------------------------------------------------------------------%
%- Functions: Injectivity, Surjectivity, Invertibility -----------------------%
%-----------------------------------------------------------------------------%
\chapter{Functions: Injectivity, Surjectivity, Invertibility}
\label{sec:Functions}

\section{Function Basics}
\label{sec:FunctionBasics}
Suppose we have the following function:
\begin{align}
  f:A \to B
  \label{eq:FunctionForm}
\end{align}
\begin{table}[!hbt]
\label{tab:ComponentsToAFunction}
\begin{tabularx}{\linewidth}{| l X |}
  \hline
  \multicolumn{2}{|l|}{Where:} \\
  \hline \hline
  $A:$ & represents the \emph{domain} (input set) \\
  $B:$ & represents the \emph{codomain} (output set) \\
  \multicolumn{2}{|l|}{The range $f \subseteq B$ is the actual output.} \\
  \hline
\end{tabularx}
\caption{Components to a function}
\end{table}
Everything in A \emph{must} be connected to everything in B. \\
\\
Example: suppose $A = \{1, 2, 3\}$ and $B = \{p,q\}$. \\
\begin{align}
     f & : A \to B \nonumber \\
  f(1) & = p \nonumber \\
  f(2) & = q \nonumber \\
  f(3) & = q \nonumber \\
\end{align}
We say the domain of $f$ is $\{1, 2, 3\}$, the co-domain $f=\{p,q\}$. range $f=\{p,q\}=B$.
This function is a little bit special because the codomain and the range are the same, but
this may not happen all the time.\\
\\
Example, suppose $A = \{1, 2, 3\}$, $B = \{a, b, \ldots, z\}$. \\
\begin{align}
  f & : A \to B \nonumber \\
  f(1) & = q \nonumber \\
  f(2) & = c \nonumber \\
  f(3) & = k \nonumber \\
\end{align}
Domain $f = \{1, 2, 3\} = A$, codomain $f = B = \{a,b,\ldots,z\}$, the range is only
$f = \{c, k, q\} \subseteq B$.\footnote{Sometimes the range is also called the \emph{image}.} \\
\\
Sometimes we say \emph{image f} to mean \emph{range f}: \\
range f = im f = $\{y \in B | f(x) = y$, for some $x \in A\} \subseteq B$. \\
\\
%TODO: right align the following comment
TODO in comments.\\
%$f: \mathbb{Z} \to {0,1}, f(x) = 0$ % 0 if x is even
%                                    % 1 if x is odd
\begin{align}
  f: \mathbb{Z} \to {0,1}, f(x) =
    \left\{
      \begin{array}{l l}
        0 & \quad \text{if $x$ is even} \\
        1 & \quad \text{if $x$ is odd} \\
      \end{array}
    \right.
\end{align}
\\
(This is sometimes called $f(x) = x$ mod $2$.)\\
\\
Domain $f = \mathbb{Z}$, codomain $f = \{0,1\}$, range $f = \{0,1\}$.\\
\\
The functions $f: A\ to B$, $g: A \to B$ are equal iff $f(x) = g(x) \forall x \in A$.\\
\\
%TODO: xypic the following
\\
% .  ==> a      . ==> a
% .  ==> b      . ==> b
% .  ==> c      . ==> c
% A      B      A     C
\\
They might look the same, but they are NOT the same because they have different codomains.
(See xypic in TODO).
\\
Another example: \\
\\
$f:$ DMTH137 $\to$ X5B.T1 \\
\\
$f(p) = s$, when $p \in $ DMTH137 is a student. \\
$s \in $ X5B.T1 is a seat. \\
\\
domain $f = \{$DMTH137 Students $\}$. \\
codomain $f = \{$seats in X5B.T1 $\}$. \\
range $f = \{$ occupied seats $\}$. \\

-----
$\phi : \mathbb{R} \to \mathbb{R}$, $\phi(x) = x^2$ \footnote{$\phi$ is greek letter phi} \\
$\phi(0) = 1$, $\phi(1) = 4$, $\phi(3) = 9$, $\phi(4) = 16$ \ldots \\

% TODO graph y = x^2
TODO: diagram \\
----- \\
What about $\psi: \mathbb{R}^{+} \to \mathbb{R}$ \footnote{$\psi$ is greek letter psi}.
\begin{align}
  \psi(x) & = \sqrt{x} && \mathbb{R}^{+} = \{x \in \mathbb{R} | x > 0 \} \nonumber \\
  \psi(1) & = \pm 1 \
  \psi(4) & = \pm 2 \
  \psi(9) & = \pm 3 \
\end{align}
This is not a function since we have 2 outputs for each input:

% TODO diagram, y = x ^2 (but rotated cw 90 deg)
TODO: diagram \\
We can make this into a function by taking only the positive square root, or only the negative square root. \\
----- \\
Suppose $|A| = n, |B| = m$, how many functions are there from A to B?
\begin{align}
  \text{Let } \{a_1, a_2, a_3, \ldots , a_n\} then \nonumber \
  f(a_1) & \text{has m possible values} \nonumber \
  f(a_2) & \text{has m possible values} \nonumber \
  f(a_3) & \text{has m possible values} \nonumber \
  \ldots \nonumber \
  f(a_n) & \text{has m possible values} \nonumber \
\end{align}
So there are $m \times m \times m \times \ldots \times m = m^n = |B|^{|A|}$ possible
functions from A to B. \\
Eg: $|A|=5, |B| = 3$,  \# $f: A \to B = 3^5 = 243$

\section{Properties of functions}
\label{sec:PropertiesOfFunctions}
A function can have several properties, namely they are
\begin{enumerate}
  \item Injectivity - a function, $f$ where the domain is a set, $A$. The
  function is \emph{injective} if for all $a$ and $b$ in $A$, if $f(a) = f(b)$,
  then $a = b$. That is to say $f(a) = f(b) \Rightarrow a = b$. By the same
  token, if $a \neq b \Rightarrow f(a) \neq f(b)$.
  
  \item Surjectivity - a function whose image is equal to it's codomain.
  $f: A \to B$ if $\forall b \in B \exists a \in A$ where $f(a) = b$. They are
  sometimes denoted by a double-headed right arrow, $f: A \twoheadrightarrow B$.
  
  \item Invertability - a function which inverts or undoes a set rather than
  individual elements from the set. $f: A \to B$, $f^{-1} B \to A$.
\end{enumerate}

\subsection{Injectivity}
Definition: a function $f: A\ to B$ is \emph{one-to-one} or \emph{injective} iff
$f(x_1) = f(x_2) \to x_1 = x_2$ (or, the y-values are unique) \\
\\
Example: \\
Consider $f:$ DMTH137 $\to$ X5B.T1. \\
f is one-to-one, otherwise people are sitting in the same seat.\footnote{which
may or may not be desirable depening on you like your seating} \\
\\
Example: \\
Consider $f: \mathbb{N} \to \mathbb{N}$, $f(x) = 2x$. \\
is $f$ one-to-one? \\
Suppose $f(x_1) = f(x_2)$, then $2x_1 = 2x_2$ so $x_1 = x_2$ hence $f$ is one-to-one. \\
 \\
Example: \\
Consider $f: \mathbb{Z} \to \{0,1\}$. $f(x) = 0$ if x is even, $f(x) = 1$ if x is odd \\

$f$ is not one-to-one $\because f(1) = f(x) = 1$ and $f(2) = f(4) = 0\ldots$ etc \\
\\
We have an infinite domain, but a finite codomain, so f cannot be 1-1. \\
\\
Example: \\
Consider $f: \mathbb{R} \to \mathbb{R}$, $f(x) = x^2$ \\
\\
Suppose $f(x_1) = f(x_2)$ then $x_1^{2} = x_2^{2}$ \\
$x_1^2 - x_2^2 = 0$ \\
$(x_1 - x_2)(x_1 + x_2) = 0$ so $x_1 = x_2 or x_1 = -x_2$ \\
hence $f$ is not one-to-one. \\

\subsection{Surjectivity}
A function $f: A \to B $ is \emph{onto} or \emph{surjective} if $\forall y \in B, \exists x \in A$
such that $y = f(x)$ \footnote{ie range f = codomain f}

% TODO: xypic this
%  .    == >  .
%  .    == >  .
%  .    == >  .
%             .
%  A          B
TODO: diagram \\
Example: $f(x) = 0$ if x is even, $=1$ if odd, $f: \mathbb{Z} \to \{0,1\}$. \\
\\
The range of $f = \{0,1\} = $ codomain $f$ \\
$\therefore$ f is surjective. \\
\\
Example: $f:$DMTH137 $\to$ X5B.T1 \\
Not surjective since there are empty seats \footnote{Fran had a sad face when she said this.}
\begin{align}
  f: A \to B \text{is onto if} \forall y \in B \\
  \exists x \in A \text{such that} y = f(x) \\
  \text{range = codomain}
\end{align}
The easiest way to decide if something is surjective is to consider the range.
Example:
\begin{align}
  f: \mathbb{R} \to \mathbb{R}, f(x) = x^2 \\
  \text{range} f = \{ y \in \mathbb{R} | y \geq 0 \} \\
  \text{but codomain} f = \mathbb{R} \\
  \therefore f \text{is not surjective}
\end{align}
On the other hand, $g: \mathbb{R} \to \{x \in \mathbb{R} | x \geq 0 \}$ then $g(x) = x^2$
is surjective. Here we have chosen the codomain to be exactly the range (image) of $f$.

\begin{align}
  f: \mathbb{Z}, f(x) = x^2 \\
  f \text{is not surjective} \because \text{range $7 \in \mathbb{N}$ but there is no $x \in \mathbb{Z}$ such that $f(x) = 7$}
\end{align}

\subsection{Invertability}
A function $f: A \to B$ is \emph{invertible} if it is \emph{both one-to-one and
surjective}. In this case, we can define an inverse function:
\begin{align}
  f^{-1} : B \to A \text{such that} x = f^{-1} = y \\
  \forall x \in A, y \subseteq B
\end{align}

%      f ==>
%  .  <===>   .
%  .  <===>   .
%  .  <===>   .
%     <== f^{-1}
%  A          B

Observe that for an invertable function $f: A \to B$ if is surjective so range $f = $ codomain $f = B$.
$x \forall y \subseteq B \exists x \in A$ with $y = f(x)$
$f$ is one-to-one, so each such $x$ is unique, ie $y = f(x)$ has a unique solution. As this is
true $\forall y \subseteq B$ then we can always find $x$ such that $y = f(x)$ \\
ie $f^{-1}$ exists.\\
\\
Example:\\
Consider $f: \mathbb{N} \to \mathbb{N}, f(x) = x$

% TODO:
% Linear graph - y = x, but only the data points from 0 to 5
See TODO:
\\
Is $f$ invertable?: \\
Is $f$ one-to-one? $f(x_1) = f(x_2) \to x_1 = x_2 \therefore$ yes. \\
Is $f$ surjective? range $f = \mathbb{N}$ Yes. \\
\\
Because $f$ is surjective and one-to-one, it is invertable.\\
\\
\\
Example: \\
Consider: $g: \mathbb{N} \to \mathbb{Z}, g(x) = x$ \\
Is $g$ is one-to-one? Yes.
Range $g = \{y \in \mathbb{Z} | y \geq 0\} = \mathbb{N}$
So $g$ is not surjective. Hence $g$ is not invertable.
\\
\\
Example:\\
Let $A = \{1,2,3,4\} = B$ \\
$f: A \to B$ \\ 
$1 \to 3$ one-to-one
$2 \to 1$ surjective since range f=3
$3 \to 4$ hence f is invertable
$4 \to 2$

$f^{-1}: B \to A$ \\
$1 \to 2$
$2 \to 4$
$3 \to 1$
$4 \to 3$

Analogy to SQL, consider the table: \\
\begin{table}[!htb]
\begin{tabularx}{\linewidth}{| c c c c c c c c c c |}
\hline
  SID      & Name    & A1 & A2 & A3 & A4 & A5 & T1 & T2 & Q \\
\hline
  41963539 & Adam    &  4 &  5 &  - &  - &  - &  - &  - & - \\
  12345678 & Michael &  5 &  4 &  - &  - &  - &  - &  - & - \\
  87654321 & Gareth  &  4 &  5 &  - &  - &  - &  - &  - & - \\
  93536914 & Tor     &  5 &  4 &  - &  - &  - &  - &  - & - \\
\hline
\end{tabularx}
\end{table}
``SELECT SID FROM DMTH137;'' \\
Represents a function on the database.\\
Let $Q:A \to B,  A = \{SID\}, B=\{A1\}$
Define $Q$ as ``SELECT SID FROM DMTH137 WHERE A1 = 5;'' \\
$B = \{0,1,2,3,3,4,5\}$ \\
$Q$ is not one-to-one, but is probably surjective as long as someone got each
of these marks. \\
\\
Suppose now we wanted to store the student data. The domain is the SID, and
the codomain is amount of space allocated to storing the data.\\

\subsection{Number of Functions on a Set}
Number of functions $f: A \to B$ is $|B|^{|A|}$. Now suppose that A and B are
finite sets.\footnote{The vertical bars represent the cardinality of the set}.\\
\\
Number of one-to-one functions:
\begin{align}
  \text{Let} |A| = m, |B| = n \\
  m > n \text{none}
  m \leq n \left\{
    \begin{array}{l l}
      \text{n choices for} & \quad f(x_1) \\
      \text{n-1 choices for} & \quad f(x_2) \\
      \text{n-2 choices for} & \quad f(x_3) \\
      \ldots
      \text{n-m choices for} & \quad f(x_m) \\
    \end{array}
    \right. \\
  n(n-1)(n-2)\ldots(n-m) = \frac{n!}{(n-m)!}
\end{align}

So the number of surjective functions:
\begin{align}
  m & < n: \text{none} & \\
  m & \geq n: \sum_{i=0}^{n}(-1)^{i}(n-i)^{n}(n-i)^{m} & \text{will be done later}
\end{align}

Number of invertable functions:
\begin{align}
  \text{We need a one-to-one} & (m \leq n )\\
  \text{and} & \nonumber \\
  \text{surjective} & (m \geq n) \\
\end{align}
The number of functions is the same as the number of ways of arranging $n$ objects
(elements)\footnote{this is called a \emph{permutation}}.

n choices for first element
n-1 choices for 2nd element
n-2 choices for 3rd element
\ldots
$n(n-1)(n-2)\ldots(2)(1) = n!$
n! invertable functions.

\subsection{Infinite sets}
For functions on infinite sets, we can still consider cardinality.

\begin{align}
  \# f: A \to B, = |B|^{|A|} \\
  f: \mathbb{Z} \to \mathbb{N} &
   & f(0) = 0 \\
   & f(1) = 1 \\
   & f(-1) = 2 \\
   & f(2) = 3 \\
   & f(-2) = 4 \\
   & \ldots
\end{align}

We can list all of $\mathbb{Z}$ in one-to-one correspondance with $\mathbb{N}$.
This means that $|\mathbb{Z}| = |\mathbb{N}|$. We can do the same sort of thing
to shoq that $|\mathbb{Q}| = |\mathbb{N}|$. We can't find a function for
$g: \mathbb{R} \to \mathbb{N}$ that is one-to-one and surjective, so
$|\mathbb{R}| \neq |\mathbb{N}|$.

\section{Functions Composites}
Suppose we were to take a set, $A$, and apply the function $f$ to form
set $B'$\footnote{using the prime notation to denote $B$ is not actually a
defined set, but a generated set}, which was then applied to function $g$:
Suppose $f: A \to B, g: B \to C$. \\
\begin{align}
  \text{Let} x \subseteq A, y \in B, z \in C \\
\end{align}

[ A ] == f ==> [ B  ] == g ==> [ C ]
  A            [ B' ]

B' is the range of f, so the domain of g has been restricted to B'.

We have $f(x) = y, g(y) = B$ \\
so $z = g(y) = g(f(x))$

The notation is as follows:  $z = g,f$ \\
and \\
$g \circ f: A \to C$ \\
We need range $f \subseteq$ domain $g$