%-----------------------------------------------------------------------------%
%- Predicate Logic and Negation
%-----------------------------------------------------------------------------%
\section{Predicate Logic and Negation}
\label{sec:PredicateLogicAndNegation}

\subsection{Predicates vs Propositions}
\label{sec:PredicatesVsPropositions}
We want to construct propositions that are either tautologies (always true) xor
contradictions (always false).

Consider the argument:
Let: \\
$p: $ All even numbers are integers\\
$q: $ 8 is even \\
$r: $ 8 is an integer \\
$(p \land q) \to r $
The problem is, this is NOT a tautology. It is inadequate in describing the
argument: \\
``All even numbers are integers and 8 is even, therefore 8 is an integer.''

Consider the argument: \\
``Not all prime numbers are odd, so there is at least one prime that is not
odd.''\\
$p: $ all primes are odd \\
$q: $ there is at least one non-odd prime \\
$-p \to q$ almost describes the argument, but is inadequate as it is not always
true.

To combat the above problems we use \emph{predicate logic}. A predicate is a
statement containing one or more variables. If values are assigned to all the
variables in a predicate the resulting statement is a
proposition.\footnote{Recall: a proposition is something we can decide the truth
value of.}

Example: $x < 5$. This is a predicate. Giving $x$ a value makes it a proposition
such as $x = 3 (\mathbb{T})$ such that $3 < 5 (\mathbb{T})$. \\
Consider the predicate $x < 5 \lor x \leq 5$ which is always true regardless of
the value of x.
We can form a proposition that is always $\mathbb{T}$:\\
For all $x$, $x<5$ or $x \geq 5$. \\
On the other hand \\
For all $x$, $x < 5 \mathbb{F}$.\footnote{If we were to leave out the
words ``for all'', the statement is sometimes true and sometimes false.}\\

There exists an $x$ such that $x < 5 \mathbb{T}$.

``For all'' and ``there exists'' are called \emph{quantifiers}. Two such
quantifiers are:
$\forall =$ ``for all''
$\exists =$ ``there exists''

Every number has a number larger than it.
Every number: $\forall x$
a number larger than x : y
$\forall x, \exists y, y>x$
For all x, there exists y, such that y > x.
% \begin{align}
%   $\forall x, (x < 5) \lor (x \geq 5)$ \\
%   $\forall x, x < 5$ \\
%   $\exists x, x < 5$
%   \intertext{We can shorten (\emph{abstract}) this notation even more:}
%   \text{Let} P(x): x < 5, Q(x): x \geq 5
%   \intertext{Then we can write a), b), c) as:}
%   \text{a)} \forall x, (P(x) \lor Q(x)) \\
%   \text{b)} \forall x, P(x) \\
%   \text{c)} \exists x, P(x) 
% \end{align}

Example: for every number x, there is another number $y$ such that $y = x + 1$.
\begin{align}
  \forall x, \exists y, y = x + 1 (\mathbb{T})
  \intertext{ or }
  \text{Let} P(x,y): y = x + 1 \text{and write} \\
  \forall x \exists y, P(x,y)
\end{align}

Eg there is a number $y$ such that for every number $x$, $y = x + 1$
\begin{align}
  \exists y \forall x, y = x + 1 (\mathbb{F})
\end{align}
These above two examples show that the order in which we write the quantifiers \emph{is important}.

Example: In a system for booking theatre seats, $B(p,s)$ is the predicate ``person $p$ has booked
seat $s$'', which in symbolic form: \\
% \begin{align}
%   \texttext{``seat s has been booked'' :} \\
%     \exists p, B(p,s) \\
%   \text{``person p has booked a seat:''} \\
%     \exists s, B(p,s) \\
%   \text{``All seats are booked:``} \\
%     \forall s, B(p,s) \\
%   \text{``No seat is double booked:`` \\ another way to write this is ``No seat is booked by
%   both person $p$ and person $q$:''} \\
%     \forall s \forall p \forall q ((B(p,s) \lor B(q,s)) \to (p \equiv q))
% \end{align}

Example: ``All swans are black.\footnote{We have some white swans, hence false}'' $\mathbb{F}$
% \begin{align}
%   \text{Let} P(x): $swan s is black$
%   \forall s, P(s) \\
%   \text{Not all swans are black} \\
%   -\forall s, P(s) \equiv \exists s, -P(s) \\
%   \text{ie, there is at least one non-black swan.} 
% \end{align}

Example: ``There is a number $x$ such that $x^{2} = 2$.
\begin{align}
  \text{Let} P(x): x^{2} & = 2 \\
  \exists x, x^{2} & = 2 (\mathbb{T})
\end{align}

``Negation: There is no number who's square is 2.'' We can easily negate it
\begin{align}
  -(\exists x, x^{2} & = 2 )
  \intertext{But we could do better, rewrite as ``ever number doesn't square to 2''}
  \forall x, -(x ^{2} = 2) & \equiv \forall x, x^{2} \neq 2 (\mathbb{F})
\end{align}

\subsection{Negation}
\label{sec:Negation}

Consider the proposition: $\forall x \exists y, y < x$ \\
Negation will \emph{invert}, \emph{reverse}, or \emph{set opposite to} and is denoted by a minus symbol
in these notes. 
\begin{align}
  -(\forall x \exists y, y < x) & \equiv \\
  & \equiv \exists x - (\forall x \exists y, y < x) \\
  & \equiv \exists x \forall y, -(y < x) \\
  & \equiv \exists x \forall y, y \geq x \mathbb{F} 
  \intertext{Which means ``there is a number $x$ that is smaller than every number $y$.'' ($\mathbb{F}$)}\end{align}

Construct the statement: ``Not all prime numbers are odd.'' ($\mathbb{T}$). \\
This means ``There is at least one prime that is not add.'' \\
\begin{align}
  \text{Let} P(x): & \text{``x is a prime number''} \\
  \text{Let} Q(x): & \text{``x is odd''}
  \intertext{To say ``all primes are odd''}
  &  \forall x, P(x) \to Q(x)
  \intertext{To say ``not all primes are odd''}
  & - (\forall x, P(x) \to Q(x) ) \\
  & \equiv \exists x, -P(x) \to Q(x)) \\
  & \equiv \exists x, -(-P(x)) \lor Q(x)) \\
  & \equiv \exists x, (--P(x) \lor -Q(x)) \\
  & \equiv \exists x, (P(x)) \lor -Q(x))
  \intertext{which means: ``There is a number $x$ such that $x$ is prime but $x$ is not odd.'' (T)} 
\end{align}

Example: Negate $\forall x \exists y(y<x) \lor (x$ is odd $)$:
\begin{align}
  -(\forall x \exists y (( y < x) \lor (x \text{is odd}))) & \\
  & \equiv \exists x, -(\exists y (( y < x) \lor (x \text{is odd}))) \\
  & \equiv \exists x \forall y -(( y < x) \lor (x \text{is odd})) \\
  & \equiv \exists x \forall y (-(y<x) \land (x \text{is odd})) \\
  & \equiv \exists x \forall y (-(y<x) \lor -(x \text{is odd})) \\
  & \equiv \exists x \forall y (( y\geq x) \lor (x \text{is even})) \text{($\mathbb{T}$)}
\end{align}

Example: Negate: $\exists x \forall y, (( y=3x) \to (y \geq x))$:
\begin{align}
  (\exists x \forall y, (( y=3x) \to (y \geq x)) ) & \equiv \\
   & \equiv - (\exists x \forall y, (( y=3x) \to (y \geq x))) \\
   & \equiv \exists x -(\forall y, (( y=3x)  \to (y \geq x))) \\
   & \equiv \exists x \forall y, (-((y=3x) \to (y \geq x))) \\
   & \equiv \exists x \forall y, (-(-(y=3x) \lor (y \geq x))) \\
   & \equiv \exists x \forall y, (( -- (=3x) \land -(y \geq x))) \\
   & \equiv \exists x \forall y, ((( y = 3x) \land -(y \geq x))) \\
   & \equiv \exists x \forall y, (( y = 3x)) \land (y<x)) \text{($\mathbb{F}$)}
\end{align}