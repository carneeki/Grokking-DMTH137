%-----------------------------------------------------------------------------%
%- Sets: Operation on Sets, Cartesian Products, Power Sets
%-----------------------------------------------------------------------------%
\section{Set Operations, Cartesian Products and Power Sets}
\label{sec:Sets}

``A \emph{set} is a bunch of stuff that we decide to lump together that we identify has some things in common.'' -- Fran Griffin.

A set is denoted by curly braces.

\begin{align}
\intertext{is not a set:}
  1, 2, 3, 4, 5 \\
\intertext{is a set:}
  \{1, 2, 3, 4, 5 \} \\
\end{align}

To say something is in a set:
\begin{align}
  x \in S \\
  \intertext{is an element of S}
\end{align}

To say something is not in a set:
\begin{align}
  x \ni S \\
\end{align}

To say something (T) is a subset of a set (S):
\begin{align}
  T \subset S \\
\end{align}

What is a subset?

Let: A, B be sets, we say  $A \subseteq B$ (A is a \emph{subset} of B) if $\forall x \in A$ we have $x \in B$
ie every element of B is also an element of A

Example:
\begin{align}
  T = \{ 1, 2, 3\}, T \subseteq S \\
  T = \{ 2, 5 \}, T \subseteq S \\
  T = \{ 3 \}, T \subseteq S \\
  T = \{ 1, 2, 4, 7 \}, T \subset S \text{since $ T \ni S $}
\end{align}
Note: $\{3\} \subseteq S $, but $ 4 \in S $

The \emph{empty set} ($\{\} = 0$) is a subset of S:
Note: $\{0\}$ is \emph{not} the empty set, as it has one element of 0.
$0 \in S, S \subseteq S$ \\ 
Note:  $0 \ni S, S \ni S$ \\
(ie, $S \in S$ is allowed, we can get into trouble with Russell's paradox)\\
\\
%[ \ldots [B] ] $ B \subseteq A $

\subsection{Equality of Sets}
\label{sec:EqualityOfSets}
$(A = B) \leftrightarrow (( A \subseteq B) \land (B \subseteq A))$

\emph{Cardinality}: ie size of a set, ie the number of elements

$|A| = $ number of elements of A.\footnote{We use the ``absolute value'' symbols
to determine the cardinality.}

\begin{align}
 |0| = 0 \\
 |S| = 5 \\
\end{align}

Combining Sets:
Let A, B be sets, $A \bigcup B$ is the set containing all the elements of both A and B.
  $x \in A \bigcup B$ if $(( x \in A )) \lor (x \in B))$
eg
\begin{align}
  S = \{ 1,2,3,4,5 \}. T = \{3, 6, 8\} \\
  S \bigcup T = \{1,2,3,4,5,6,9\} \\
\end{align}
Note: expanded elements in a union are listed only once. Order is not important, eg:
\begin{align}
  \{1,2,3\} = \{2, 3, 1\} \\
\end{align}

Definition: Let A, B, be sets $A \bigcap B$ (read as ``A intersect B'') is the set
of elements common to both A and B.\\
\\
$x \in A \bigcap B$ if $(( x \in A) \land (x \in B))$. \\
eg \\
$ S \{1,2,3,4,5\}, T = \{2, 4, 6, 8\}$ \\
$ S \bigcap = \{2, 4\} $ \\
% TODO: venn diagram of A \bigcap B
% TODO: venn diagram of A \bigcup B

$|A \bigcup B | = |A| + |B| - |A \bigcap B|$ \\

\subsection{Universal Set}

$\mathbb{N} = \{ 1, 2, 3, \ldots\} $ Natural Numbers.\footnote{Mathematicians can't decide whether zero is included or not}. \\
$\mathbb{Z} = \{ -3, -2, -1, 0, 1, 2, 3 \} $ Integers. \\
$\mathbb{Q} = \{ \frac{a}{b} | a, b, \in Z, b \neq 0$ Rational Numbers \\
$\mathbb{R} = \{$ numbers that can be written as decimals $\}$. Real numbers. \\

$\pi$, $\emph{e}$, etc\ldots are not rational, but are real.

$\mathbb{N} \subseteq \mathbb{Z} \subseteq \mathbb{Q} \subseteq \mathbb{R} $ \\

$|\mathbb{N}| = \infty$
$|\mathbb{Z}| = \infty$
$|\mathbb{Q}| = \infty$
$|\mathbb{R}| = \infty$

$|\mathbb{N}| = |\mathbb{Z}| = |\mathbb{Q}| < |\mathbb{R}|$ \\
We can list the elements of $\mathbb{N}$ without missing any, 1, 2, 3, \ldots \\
We can list the elements of $\mathbb{Z}$ in correspondance with elements of $\mathbb{N}:$

% TODO: TeXorize this table:
%
% N: 1, 2,  3,  4,  5,  6,  7,  8,   9,  10, 11
% Z: 0, 1, -1,  2, -2,  3, -3,  4,  -4,   5, -5
%

So the sizes of $\mathbb{N}$ and $\mathbb{Z}$ are the same. We can do this same for $\mathbb{Q}$
(as an exercise). We cannot list the elements of $\mathbb{R}$ in correspondance with $\mathbb{N}$.

To see this, suppose we have a list of all real numbers ($\mathbb{R}$):
(where a, b, c, d, e, \ldots are decimal digits)
%0 . a_1 b_1 c_1 d_1 e_1 \\ 
%0 . a_2 b_2 c_2 d_2 e_2 \\
%0 . a_3 b_3 c_3 d_3 e_3 \\
%0 . a_4 b_4 c_4 d_4 e_4 \\
%0 . a_5 b_5 c_5 d_5 e_5 \\
% TODO: vertical 3dots
But if we take one digit from each number in our list and change it then we have:
%0. a_1^1 b_2^1 c_3^1 d_4^1 e_5^1

Which differs form the $i^{th}$ number in my list at the $i^{th}$ digit. Hence, this
number was not in the list.

$|\mathbb{N}| = |\mathbb{Z}| = |\mathbb{Q}|$ is called \emph{``countable
infinity''}. $|R|$ is \emph{``uncountable infinity''}.

Universal Set $\epsilon$: \\
Let $\epsilon = \mathbb{N}$ \\
Let $S = \{1,2,3,4,5\}$ \\
The complement of S:
%TODO: S-bar
$ -S = \{ x \in \mathbb{N} | x \ni S \} $

%TODO: venn diagram of epsilon and S-bar

$S = \{1,2,3,4,5\}$, $T = \{1,2\}$ \\
$S \ T=\{3, 4, 5\}$ or $S - T = \{3, 4, 5 \}$ is read as ``complement of T is S'' \\
$|S - T| = |S| - |S \bigcap T|$ \\
eg \\
$A = \{ x \subseteq \mathbb{Z} | x \geq 5 \}, B \{ x \subseteq \mathbb{Z} | x < 8 $
$|A| = |B| = \infty$
$|A \bigcap B | \{ x \subseteq \mathbb{Z} | (x \geq 5) \land (x < 8)\} = \{5,6,7\} $ 
