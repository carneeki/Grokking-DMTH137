%-----------------------------------------------------------------------------%
% Format and styling in this file originally created by 
% Carl E. Svensson 2010, updated by Adam J. Carmichael 2011
%-----------------------------------------------------------------------------%
%
%- Document Makeup -----------------------------------------------------------%
%- (01) Notes from template author
%- (02) Document Class and Options
%- (03) Standard package includes and options
%- (04) Custom Definitions and Alterations
%- (05) Custom Commands
%- (06) Document title and other metadata
%- (07) Start Document Content
%- (07a) Misc Config 
%
%
%- (01) Notes from carneeki@ -------------------------------------------------%
% NEATNESS:
%  Please keep the TeX neat. Best ways to do this:
%  (01) Don't indent
%  (02) Keep inside of 80 characters (it makes for nicer editing on small
%       laptops).
%  (03) Avoid whitespace between \section{} and other document elements. We
%       have %%%% comments for a reason!
%  (04) Use 2 (that's TWO) space characters to indent. NEVER use tab unless
%       your editor converts to to space chars.
%  (05) Maintain customisations in their respective sections.
%  (06) Comment everything. Bandwidth and diskspace are cheap these days, and
%       TeX compresses pretty nice. Anything else is the BAD kind of laziness
%       on your part.
%
% MULTILINE EQUATIONS:
%  Use \begin{align} instead of \begin{eqnarray}...
%  Details as to why are found at (tl;dr : it's just better...):
%  http://texblog.net/latex-archive/maths/eqnarray-align-environment/
% 
% BIBLIOGRAPHY: 
%  -> URLS: to generate the GUID for a reference that is for a URL, paste
%     the URL into goo.gl and then take only the suffix portion.
%
%  -> Wikipedia citations, simply copy + paste the citation from the
%     menu on the LHS.
%
%
%- (02) Document Class and Options -------------------------------------------%
\documentclass[
%  pagesize,
  a4paper,
  pdftex,
%  fontsize=11pt,
  draft=false,
  twoside,
]{book}
%
%
%- (03) Standard package includes and options --------------------------------%
%\usepackage{draftwatermark} % draft watermark. Comment these 2 lines in final
%\SetWatermarkLightness{0.9} %
\usepackage{amsmath}       % amsmath & amssymb are almost ALWAYS required.
\usepackage{amssymb}       %
%
%\usepackage{verbatim}     % multiline commenting ( c++ equiv /* ... */ )
%
\usepackage{xcolor}         % pdflatex
\definecolor{neekiRed}{RGB}{172,40,41}
\definecolor{neekiBlue}{RGB}{62,70,157}
%
\usepackage{geometry}      % option for altering page dimensions if needed
\usepackage[pdftex]{graphicx} % including image files for figures (ie
                              % non-[E]PS)
%                          % valid types: jpeg, png, pdf
\usepackage{wrapfig}       % the figures themselves
\usepackage[numbers,
square,
longnamesfirst
]{natbib}                  % prettybib
\usepackage[pdftex]{hyperref} % clickable TOC and refs
%\usepackage[all]{xy}      % category theory helpers
%\xyoption{all}            % category theory helpers
%\input xy                 % category theory helpers
\usepackage{tikz}        % easy graphic thing
\usepackage{tabularx}    % easy tables
\usepackage{url}         % easy urls
\usepackage{multirow}    % 
\usepackage{lipsum}      % autogen placeholder text
%
%- (04) Custom Definitions and Alterations -----------------------------------%
\usepackage[T1]{fontenc} % font-doohickey
%\usepackage{tgadventor}  % font
%\usepackage[math]{iwona} % font
\usepackage[light,math]{kurier} % font
%
\linespread{1.5} % carneeki@ use approx 150% line spacing just like MaxDesign.
\hypersetup{
  % DO NOT CHANGE THESE
%   pdftitle={\metaTitle},
%   pdfauthor={\metaAuthorShort},
%   pdfsubject={\metaSubject},
%   pdfkeywords={\metaKeywords},
  pdfcreator={LaTeX},
  pdfproducer={LaTeX},
  pdftoolbar=false,
  % But change these to taste:
  pdffitwindow=false,   % window fit to page when opened
  pdfstartview={Fit},   % fits the width of the page to the window
                        % all (useful) opts: Fit, FitH, FitV,
  pdfnewwindow=true,    % open links in new window
  colorlinks=true,      % false = boxed links; true = colored links
  linkcolor=neekiRed, % color of internal links
  citecolor=neekiBlue, % color of links to bibliography
  filecolor=red, % color of file links
  urlcolor=red  % color of external links
}
%
%- (05) Custom Commands ------------------------------------------------------%
\newcommand{\derivative}[1][x]{\frac{\mathrm{d}}{\mathrm{d}#1}}
\newcommand{\lxor}{\oplus}
\newcommand{\lxnor}{\lnot\lxor}
%
%- (06) Document Title and other metadata ------------------------------------%
%
\title{
  Grokking DMTH137
}
%
\author{
  Adam J. Carmichael \\
  Undergraduate Student \\
  Department of Electronic   Engineering\\
  Macquarie University\\
  Sydney, Australia 2109\\
  Email: \url(adam.carmichael@ieee.org) \\
} % author END Brace
%
%- (07) Start Document Content -----------------------------------------------%
\begin{document}
%- (07a) Misc Config ---------------------------------------------------------%
%-----------------------------------------------------------------------------%
%\cfoot{\thepage\ of \pageref{LastPage}} % page n of m
%
\maketitle
%
%\begin{abstract}
% This document forms the notes for DMTH137 by the author. DMTH137 changed from
% programming in C++ to programming in Java however the essence of DMTH137
% syllabus the same. It is about linked data structures. 
%\end{abstract}
%
\section{Introduction}
\label{sec:Introduction}
DMTH137 is about data structures and algorithms.
%-----------------------------------------------------------------------------%
%- Table of Contents ---------------------------------------------------------%
%-----------------------------------------------------------------------------%
\tableofcontents
%
\newpage

%-----------------------------------------------------------------------------%
%- Networks ------------------------------------------------------------------%
%-----------------------------------------------------------------------------%
\chapter{Networks}
\label{chap:Networks}
%-----------------------------------------------------------------------------%
%- Networks :: Undirected and Directed Graphs --------------------------------%
%-----------------------------------------------------------------------------%
\section{Undirected and Directed Graphs}
\label{sec:UndirectedAndDirectedGraphs}
\lipsum[1]

%-----------------------------------------------------------------------------%
%- Networks :: Spanning Trees and Traversal Strategies -----------------------%
%-----------------------------------------------------------------------------%
\section{Spanning Trees and Traversal Strategies}
\label{chap:SpanningTreesAndTraversalStrategies}
%-----------------------------------------------------------------------------%
%- Networks :: Walks, Paths and Cycles ---------------------------------------%
%-----------------------------------------------------------------------------%
\section{Walks, Paths and Cycles}
\label{sec:WalksPathsCycles}
\lipsum[1]

%-----------------------------------------------------------------------------%
%- Networks :: Direct Proofs and Proofs By Contradictions --------------------%
%-----------------------------------------------------------------------------%
\section{Object Oriented Design and Development}
\label{chap:ObjectOrientedDesignandDevelopment}
\lipsum[1]

%-----------------------------------------------------------------------------%
%- Networks :: Modular Arithmetic --------------------------------------------%
%-----------------------------------------------------------------------------%
\section{Modular Arithmetic}
\label{chap:ModularArithmetic}

%-----------------------------------------------------------------------------%
%- The Natural Numbers and Mathematical Induction ----------------------------%
%-----------------------------------------------------------------------------%
\section{The Natural Numbers and Mathematical Induction}
\label{chap:NaturalNumbersAndMathematicalInduction}

%-----------------------------------------------------------------------------%
%- Prime Numbers and Factorisation -------------------------------------------%
%-----------------------------------------------------------------------------%
\section{Prime Numbers and Factorisation}
\label{sec:PrimeNumbersAndFactorisation}

%-----------------------------------------------------------------------------%
%- Greatest Common Divisors and Least Common Multiples -----------------------%
%-----------------------------------------------------------------------------%
\section{Greatest Common Divisors and Least Common Multiples}
\label{sec:GCDandLCM}

%-----------------------------------------------------------------------------%
%- Euclid's Algorithm
%-----------------------------------------------------------------------------%
\section{Euclid's Algorithm}
\label{sec:EuclidsAlgorithm}

%-----------------------------------------------------------------------------%
%- Chinese Remainder Theorem
%-----------------------------------------------------------------------------%

%-----------------------------------------------------------------------------%
%- Logic
%-----------------------------------------------------------------------------%
\chapter{Logic}
\label{chap:Review}

%-----------------------------------------------------------------------------%
%- Logic :: Propositional Logic and Truth Tables
%-----------------------------------------------------------------------------%
\section{Propositional Logic and Truth Tables}
\label{sec:PropositionalLogicAndTruthTables}

A proposition is a statement that is either true or false.

``It is Thursday'' is a statement, which may be true or false depending on the
day.
``Pigs can fly'' is a statement, which is most likely false unless something odd
is done to the pig.
$2 + 3 = 5$ is a statement which turns out to be true (T).
$2 + 3 = 7$ is a statement which turns out to be false (F).
$ -3 < 7  $ (T)
$\frac{5}{4} > \frac{6}{5}$ (T)

Some things which are not propositions:
\begin{itemize}
  \item ``Go away.'' is a statement, but cannot be resolved to a true or false
  value. ``Because it's Monday''. This is only a sentence fragment 
  \item $ x + y $
  \item ``I am lying.'' is a paradox. If you actually are lying, then the
  statement is true and thus no longer a lie (and hence the truth). Paradoxes come about due to
the notion of ``self-reference''.
  \item ``This sentence is false'' is another example. Interestingly though,
  ``This sentence is true'' is not a paradox.
  \item ``Nothing is true'' is another paradox \footnote{and is an oft used
  topic in post-modernism, but that's arts topic and best left to people who
  belong in Y3A}.
  \item Russell's Paradox
\end{itemize}

Most things that we want to say contain more than a single idea so we must
combine, so we must combine propositions in some way.

Propositions are often represented by the letters p and q. They can be combined
using things called ``connectives'' like:
\begin{itemize}
  \item and
  \item or
  \item not
  \item xor
  \item xnor
  \item implies
  \item iff (if and only if)
  \item equivalent
\end{itemize}

Let ``p'' be the statement ``I like tea'' and ``q'' be the statement ``I like
coffee''.

If I wanted to say: ``I like tea and coffee'': $p \land q$
If I wanted to say: ``I like tea or coffee'': $p \lor q$
If I wanted to say: ``I like tea but not coffee'': $p \land \lnot q$
``I don't like tea or coffee'' can be written two ways: $ -(p \lor q)$ or $\lnot p
\land \lnot q $

Let ``p'' be the statement ``It is raining.''
Let ``q'' be the statement ``I need an umbrella.''

If it is raining, then I need an umbrella:

$ p \to q $

ie, the fact that it is raining implies that I need an umbrella.

If it is not raining, so I don't need an umbrella.

$ \lnot p \to \lnot q $

What about $q \to p$?
This says: ``If I need an umbrella, then it is raining.'' This is not
necessarily true as there may be many uses for an umbrella besides keeping rain
off\footnote{eg an umbrella salesman or as a parachute}.

Let ``p'' be the statement ``he is vegetarian''
Let ``q'' be the statement ``he eats no meat''

He is vegetarian if and only if he eats no meat.

$ p \iff q \equiv (p -> q) \land (q -> p)$
This establishes equivalence between proposition p and q.


Conjunction $p \land q$ is true only when both p,q are true.
\begin{table}[!htb]
\label{tab:TruthTableAND}
\begin{tabularx}{\linewidth}{| c | c | X |} \hline
  p & q & $(p \land q)$ \\ \hline \hline
  F & F & F \\ \hline
  F & T & F \\ \hline
  T & F & F \\ \hline
  T & T & T \\ \hline
\end{tabularx}
\caption{Truth Table: logical AND}
\end{table}

Disjunction $p \lor q$ is true when at least one of p,q are true:
\begin{table}[!htb]
\label{tab:TruthTableOR}
\begin{tabularx}{\linewidth}{| c | c | X |} \hline
  p & q & $(p \lor q)$ \\ \hline \hline
  F & F & F \\ \hline
  F & T & T \\ \hline
  T & F & T \\ \hline
  T & T & T \\ \hline
\end{tabularx}
\caption{Truth Table: logical OR}
\end{table}

Negation $\lnot p$ is true when $p$ is false
\begin{table}[!htb]
\label{tab:TruthTableNOT}
\begin{tabularx}{\linewidth}{| c | X |} \hline
  p & $(\lnot p)$ \\ \hline \hline
  F & T \\ \hline
  T & F\\ \hline
\end{tabularx}
\caption{Truth Table: logical NOT}
\end{table}

Conditional $p \to q$ if p then q, or ``p implies q''. This is false when
$p$ is true and $q$ is false with a default condition of true.

\begin{table}[!htb]
\label{tab:TruthTableIMPLIES}
\begin{tabularx}{\linewidth}{| c | c | X |} \hline
  p & q & $(p \to q)$ \\ \hline \hline
  F & F & T \\ \hline
  F & T & T \\ \hline
  T & F & F \\ \hline
  T & T & T \\ \hline
\end{tabularx}
\caption{Truth Table: logical IMPLIES}
\end{table}

A plain English example: It can't rain when there are no clouds, so ``if it is
raining then it is not cloudy'' is false.

Suppose it isn't raining, it could still be cloudy. This remains true.

Converse $q \to p$
\begin{table}[!htb]
\label{tab:TruthTableCONVERSE}
\begin{tabularx}{\linewidth}{| c | c | X |} \hline
  p & q & $(q \to p)$ \\ \hline \hline
  F & F & - \\ \hline
  F & T & - \\ \hline
  T & F & - \\ \hline
  T & T & - \\ \hline
\end{tabularx}
\caption{Truth Table: logical CONVERSE}
\end{table}

If it is cloudy then it is raining (not always true).
q->p is NOT the same as $p -> q$

Contrapositive $\lnot p -> \lnot q$
Use to establish the notion of ``proof by contradiction''.

Double conditional $ p \equiv q $
p if and only if q: p iff q
$p \to q$ and $q \to p$ must both be true
That is $(p \equiv q) \equiv (( p \to q) \land (q \to p))$

\begin{table}[!htb]
\label{tab:TruthTableCONVERSE}
\begin{tabularx}{\linewidth}{| c | c | X |} \hline
  p & q & $( p \equiv q ) \equiv (( p \to q) \land (q \to p))$
                                                    \\ \hline \hline
  F & F & F \\ \hline
  F & T & F \\ \hline
  T & F & F \\ \hline
  T & T & T \\ \hline
\end{tabularx}
\caption{Truth Table: logical DOUBLE CONDITIONAL}
\end{table}

eg let p: 4 is an even number
       q: 4 is divisible by 2
       p and q clearly mean the same thing.
       
Exclusive or (xor) $p \lxor q$
This is true when either p,q is true, but not both.

\begin{table}[!htb]
\label{tab:TruthTableXOR}
\begin{tabularx}{\linewidth}{| c | c | X |} \hline
  p & q & $( p \lxor q )$ \\ \hline \hline
  F & F & F \\ \hline
  F & T & T \\ \hline
  T & F & T \\ \hline
  T & T & F \\ \hline
\end{tabularx}
\caption{Truth Table: logical XOR}
\end{table}

How large should you make your truth table?
\begin{enumerate}
  \item For every variable we have 2 possible values: True and False.
  \item If we have n propositions we need ${2}^{n}$ rows in the truth table.
\end{enumerate}
For example, if we have the variables $p,q,r$ then we need ${2}^{3} = 8$ rows.

% TODO: fix order of operations

Order of operations:

\begin{enumerate}
  \item Do OR connectives
  \item Do AND connectives
\end{enumerate}


Two expressions are logically equivalent if their truth values match for every
combination of the truth values of their atomic propositions\footnote{an atomic
proposition is a proposition cannot be broken down any further}. This is the
same as saying the expressions have the same truth tables.

We want to show $p \land (q \lor r) \equiv ( (p \land q) \lor (p \land r) )$:
\begin{enumerate}
  \item Build truth table. We need 8 rows.
\end{enumerate}

Tautology: In ordinary language it is when you use redundant words which are
unnecessary \footnote{This explanation in itself is a tautology where redundant
and unnecessary mean the same thing}. A tautology is always true. The opposite
of a tautology is a contradiction; which is always false.

Tautology example: $p <-> q $ iff 

\begin{table}[!htb]
\label{tab:TruthTableTautology}
\begin{tabularx}{\linewidth}{| c | c | c | c | X |} \hline
  $p$ & $q$ & $(p \land q) $ & $ \equiv $ & $ (\lnot p \lor \lnot q) $
                                                      \\ \hline \hline
   F  &  F  &               &       &                 \\ \hline
   F  &  F  &               &       &                 \\ \hline
   F  &  F  &               &       &                 \\ \hline
   F  &  F  &               &       &                 \\ \hline
\end{tabularx}
\caption{Truth Table: Tautology}
\end{table}


%-----------------------------------------------------------------------------%
%- Laws of Logic
%-----------------------------------------------------------------------------%
\section{Laws of Logic}
\label{sec:LawsOfLogic}

%-----------------------------------------------------------------------------%
%- Predicate Logic and Negation
%-----------------------------------------------------------------------------%
\section{Predicate Logic and Negation}
\label{sec:PredicateLogicAndNegation}

%-----------------------------------------------------------------------------%
%- Sets: Operation on Sets, Cartesian Products, Power Sets
%-----------------------------------------------------------------------------%
\section{Set Operations, Cartesian Products and Power Sets}
\label{sec:Sets}

%-----------------------------------------------------------------------------%
%- Relations: Symmetry, Reflexivity, Transitivity, Equivalence
%-----------------------------------------------------------------------------%
\section{Relations: Symmetry, Reflexivity, Transitivity, Equivalence}
\label{sec:Relations}

%-----------------------------------------------------------------------------%
%- Functions: Injectivity, Surjectivity, Invertibility
%-----------------------------------------------------------------------------%
\section{Functions: Injectivity, Surjectivity, Invertibility}
\label{sec:Functions}

%-----------------------------------------------------------------------------%
%- Combinatorics: Counting Arguments, Permutations and Combinations
%-----------------------------------------------------------------------------%
\section{Combinatorics: Counting Arguments, Permutations and Combinations}
\label{sec:Combinatorics}

%-----------------------------------------------------------------------------%
%- Principle of Inclusion-Exclusion
%-----------------------------------------------------------------------------%
\section{Principle of Inclusion-Exclusion}
\label{sec:InclusionExclusion}

%-----------------------------------------------------------------------------%
%- The Binomial Theorem and Extended Binomial Theorem
%-----------------------------------------------------------------------------%
\section{Binomial Theorem and Extended Binomial Theorem}
\label{sec:BinomialTheorem}

%-----------------------------------------------------------------------------%
%- Boolean Algebra
%-----------------------------------------------------------------------------%
\section{Boolean Algebra}
\label{sec:BooleanAlgebra}

%-----------------------------------------------------------------------------%
%- Logic Gates
%-----------------------------------------------------------------------------%
\section{Logic Gates}
\label{sec:LogicGates}

%-----------------------------------------------------------------------------%
%- Minimisation of Digital Circuits
%-----------------------------------------------------------------------------%
\section{Minimisation of Digital Circuits}


%-----------------------------------------------------------------------------%
%- Acknowledgment ------------------------------------------------------------%
%-----------------------------------------------------------------------------%
\newpage
\section{Acknowledgements}
\label{sec:Acknowledgements}
I (Adam) had a whole swag of people to help me along the way. Listed, in no
particular order (because there is no fair way to list them), they are:
\begin{itemize}
  \item Carl Svensson, Macquarie University, for the \LaTeX, the maths, and the
  many late night sessions over a family dinner box, and the many in jokes and
  innuendoes\footnote{Giggity}.
  \item Michael Griffin, Macquarie University, for proof reading and finding
  errors.
  \item Josh Larietti, Macquarie University, for more maths.
  \item Celeste Cohen, for letting me show off stuff to her that I thought
  was pretty cool, whilst being completely irrelevant. For advice on page
  layout and wording. For being a friend when I needed one. For everything.
  \item The Heimlich Family, Macquarie University, for giving me a fantastic
  opportunity to put things I've learned into practice, and for the learning
  that resulted from it. To Mike, Luan, Sarah and Jaye for making FIRST happen
  in Australia, and for inviting me to become a part of it.
  \item FIRST Team 3132, The Thunder Down Under, for always holding me to high
  standards of Gracious Professionalism\texttrademark\footnote{Gracious
  Professionalism is a common law trademark of the United States Foundation for
  Inspiration and Recognition of Science and Technology (US FIRST).}
  \item Mark Leon, NASA, for the words of inspiration and wisdom when you
  spoke at the 2011 Honolulu FIRST FRC regionals. \quote{\ldots at the end of
  the day, it will be the engineers who save the world. This is why we do the
  math\ldots}
  \item Engineering \& math staff at Macquarie, David Wong, Yinan Kong, Sam
  Reisenfeld, Tony Parker, Rein Vaseilo, Barry McDonald. You make engineering
  awesome!
  \item It would be remiss of me to not mention the pit crew who make
  sure that I keep going lap after lap\ldots Nathan, Nick, Diana, Heidi, Hugh,
  Jessica, Frankie, VK2BV, Stephen VK2TQ, Will, Pippa, David, Emily, Andrew,
  John, Sue, Matthew, Richard, my brother Sean, and my mother and father.
\end{itemize}
% References
% Bibliography
\bibliography{DMTH137}
\bibliographystyle{abbrvnat}
%
\end{document}