%-----------------------------------------------------------------------------%
%- Networks :: Undirected and Directed Graphs --------------------------------%
%-----------------------------------------------------------------------------%
\chapter{Undirected and Directed Graphs}
\label{chap:UndirectedAndDirectedGraphs}
We are used to graphs are often used to describe functions such as $y = f(x)$.
We will not be using them in DMTH137.

It is important to note that a graph is a collection of points called
``\emph{vertices}'' and lines or connections which join them called
``\emph{edges}''.
A vertex can represent a piece of information and the edge represents how they
are connected.
Example: A vertex represents people and the edges represent the degrees of
separation to which they know eachother.
\\
Example: Each vertex represents a town and the edge represents the distance
between each town.
\\
\emph{Edges must join two distinct vertices}. A graph with loops allows an edge
to join a vertex to itself. (If you have had Cisco switching experience: compare to the
packet storm created in a switched network without spanning tree protocol
implemented).
\\
There are two types of graphs we look at in DMTH137:
\begin{enumerate}
  \item Directed graphs
  \item Undirected graphs
\end{enumerate}

A directed graph has arrows along the edges to indicate the flow of the
direction. Undirected graphs have no arrows. An arrow may point one way, the
opposite way, or double arrows to indicate bidirectionality.

% TODO: include example of directed graph ( ->, <-, <-> )

% TODO: include example of undirected graph 

Sometimes a graph (be it directed or undirected) will have a number or
\emph{weight} with an edge, and is written next to the edge itself.

% TODO: include example of weighted graph.

In terms of language, we say ``Vertex A is adjacent to Vertex B if there is an
edge from A to B''.

\section{Equivalence in Graphs}
How do we know when two graphs are the same graph?
\\
When the vertices in one graph, GraphA are able to be translated to a second
graph, GraphB such that both sets of vertices are joined in the same way.
$ A-> (B,C,D) $ % TODO: let this be a messy graph of spaghetti.
$ 1-> (2,3,4) $ % TODO: let this one look like Bruce Schneir's alphabet soup
                % (through his eyes)

There is only one graph with a single vertex:
% TODO: draw a single vertex
\begin{tikzpicture}
\end{tikzpicture}


2 Vertices:
% TODO: draw 2 points
% TODO: draw 2 points connected

3 vertices:
% TODO: draw 3 points
% TODO: draw 3 points with 2 points connected
% TODO: draw 3 points with 3 points connected

4 vertices:
% TODO: draw 4 points
% TODO: draw 4 points with 2 points connected
% TODO: draw 4 points with 3 points connected 
  % TODO: draw 4 points with 3/4 box
  % TODO: draw 
% TODO: draw 4 points with 4 points connected
  % TODO: draw 4 points as box
  % TODO: draw 4 points with 3/4 box and 1 diagonal

% TODO: just refer to the photos...

\section{Complete graph}

$K_n$ - the complete graph of n vertices is the graph of n vertices
where every pair of vertices is adjacent

$K_3$ (3 points; draw a triangle); 3 edges
$K_4$ (4 points; draw a square and a bend); 5 edges
$K_5$ (5 points; draw a pentagram in a pentagon); 10 edges

\section{Bipartite graph}
A bipartite graph is a graph where the vertices are arranged into two groups or
areas.
% TODO: refer to photos
$K_(n,m)$

$K_(3,4)$ (a graph in 2 parts) % bipartite

$K_(3,2)$

\section{Planar Graphs - Euler's Formulae}
A graph is said to be \emph{planar} if it can be drawn on a plane so that edges
do not cross. The formula to determine if a graph is planar or not is denoted by
\begin{align}
  V + F - E & = 1
\end{align}
Where $V$ is the number of vertices, $F$ is the number of faces, and $E$ is the
number of edges.
\\
A \emph{face} in a graph is essentially a circuit between arbitrary vertices.
\\
If a graph can be drawn on a sphere where it's edges do not cross, we can use
the formula to partly determine if it is spherical planar:
\begin{align}
  V + F - E & = 2
\end{align}
The second step is to compare the average number of edges per face and compare
it to the smallest number of edges per face:
\begin{align}
  \frac{2E}{F} & = \text{avg number of faces}
\end{align}
If the average number of faces is less than the smallest number of faces, then
the graph is not spherical planar.

The final step is to examine the girth. The \emph{girth} of a graph is the
shortest circuit. If a connected graph has V vertices, E edges and a girth of $g$ and
\begin{align}
  \frac{2E}{2+E-v} < g
\end{align}
Then the graph is not planar.