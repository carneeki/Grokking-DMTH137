%-----------------------------------------------------------------------------%
%- Greatest Common Divisors and Least Common Multiples -----------------------%
%-----------------------------------------------------------------------------%
\chapter{Greatest Common Divisors and Least Common Multiples}
\label{chap:GreatestcommonDivisorAndLeastCommoonMultiples}

\section{Divisibility}
\label{sec:Divisibility}
We say that ``$a$ divides $b$'' or ``$b$ is divisible by $a$'' if $b$ is a
multiple of $a$. The symbol to represent this we use a vertical bar.
\begin{align}
   a|&b  &~ \text{``a divides b''} & \\
   5∤&11 &~ \text{``5 does not divide 11''} & \\
  27|&0  &~ \text{``27 divides 0. (Think about $\frac{0}{27}$)''} & \\
   0|&0  &~ \text{``0 divides 0''} & \\
   0∤&6
\end{align}
Amongst the natural numbers $\mathbb{N} = \{0, 1, 2, 3, \ldots\}$, 0, 1 are
special. For the others, they are either prime or composite. \emph{Primes} are
divisible by exactly 2 numbers, themselves and 1. \emph{Composittes} are
divisible by numbers other than 1 and themselves.\footnote{1 is neither
composite nor prime. It is called \emph{the unit}.}

Example:\\
$\{2, 5, 7, 11, 13, 17, 19, 23 \}$ are all prime numers
$\{0,4,6,8,9,10,12,14,15,16,18,20,21,22\}$ are all composite numbers\\
\\
How many prime numbers are there? Infinitely many.\\
Proof:
\begin{align}
  \text{Suppose there were only finitely many primes:} \\
  S = \{P_1, P_2, P_3, \ldots P_n \} \\
  \text{Consider} \\
  P_1 \times P_2 \times P_3 \ldots \times P_n + 1 & = \ldots \\
   & = \ni S \\
  \text{Is not in S, so it cannot be a prime. So it must have a prime factor.
  But, none of the primes divide it, because they all leave a remainder of 1.
  So it cannot be composite either.} \\
  & \therefore \text{a contradiction.} \\
  & \Rightarrow \text{the set of primes must be infinite.}
\end{align}

The probability of a large number being prime is approximately $\frac{1}{log_n}$.

\section{Common Factors \& Greatest Common Factor}
What are the common factors of $28, 164$? \\
\\
Factors of $28 = \{ 1,2,4,7,14,28 \}$ \\
Factors of $164 = \{1, 2, 4, 41, 82, 164 \}$ \\
\\
The common factors are $\{1,2,4\}$ and the greatest common factor is $4$. \\
\\
What is the greatest common factor of $1234$ and $5762$?\\
\\
Doing many divisions is computationally costly and time consuming. We need a
better way to make this computation feasible.\\
\\
By using prime factors (ie factors of a composite number which are prime), we
can determine composite factors much faster:
\begin{align}
  164 & = 2 * 82 \\
      & = 2*2*41 \\
      & = 2^2 * 41 \\
\end{align} 
or
\begin{align}
  2^0 * 41^0 & = 1 \\
  2^1 * 41^0 & = 2 \\
  2^2 * 41^0 & = 4 \\
  2^0 * 41^1 & = 41 \\
  2^1 * 41^1 & = 82 \\
  2^2 * 41^1 & = 164
\end{align}
So $3^2 * 5^3 * 7^4$ has $3*4*5 = 60$ facotrs\\
\\
So factoring both numbers into products of primes will let us find the factors
and common factors easily and so might be a good way to go.\\
\\
GCF of $3^2 * 11^4 * 17^3$ is $8! = 1*2*3*4*5*6*7$ is easily worked out:
\begin{align}
  8! & = 2* 3 * 2^2 * 5 * 7 \\
     & = 2^4 * 3^2 * 5 * 7
\end{align}
So the greatest common factor is $3^2 = 9$.\\
\\
GCF of $2^5 * 3^2 * 11^4 * 17^3$ is $8! = 1*2*3*4*5*6*7$ is easily worked out:
\begin{align}
  8! & = 2* 3 * 2^2 * 5 * *2 * 3* 7 \\
     & = 2^4 * 3^2 * 5 * 7
\end{align}
So the greatest common factor is $2^4 * 3^2 = 9 * 18 = 144$.\\
\\
However, this is still computationally unfeasable for large numbers as it is
prohibitively time consuming so this method is also impractical, but, the
ancient Greeks knew how to do it easily.\footnote{This ancient Greek dude called
Euclid wrote a computer program in 300BC, but didn't have a PC to run it on so
he used a sandbox.}

\section{Euclid's Algorithm}
\label{sec:EuclidsAlgorithm}