%-----------------------------------------------------------------------------%
%- Networks :: Direct Proofs and Proofs By Contradictions --------------------%
%-----------------------------------------------------------------------------%
\chapter{Proofs}
\label{chap:Proofs}

\section{Proof by Induction}
\quotation{``Mathematical induction proves that we can climb as high as we like
on a ladder, by proving that we can climb onto the bottom rung (the basis) and
that from each rung we can climb up to the next one (the induction)'' -- Donald
Knuth, Concrete Mathematics}
Question: How can we prove an infinite number of things? ie How can you prove
a formula is always true? \\
The definition of $\mathbb{N}$ comes to our rescue; it includes the statement:
if $S \in \mathbb{N}$ and $0 \in S$ and $\forall n \in S \Rightarrow (n + 1) \in S $

Essentially, if you prove a case for $n$, and the statement remains true
for $n+1$, then you have formulated a proof by induction.

\section{Proof by Strong induction}
\label{sec:StrongInduction}
Strong induction says that we may assume not only that the original statement
holds, but is true for all cases up to a certain boundary point.\footnote{It's
sort of like turbo-charged version of induction, it works similarly to
induction, but proves a wider scope and specifies a boundary.}

If you prove a case for $n$ and you prove it for \emph{all} $n$ values up to and
including $m$, then you have formulated a proof by strong induction.

Let $S \in \mathbb{Z}$ and suppose that $0 \in S$ and $(forall n \in \mathbb{Z}, (m < n) \Rightarrow n \in S) \Rightarrow n \in S$ \\
\\
Then $S = N$ \\
\\
Example: Prove that every positive integer $> 2$ is the product of primes of 1
or more prime numbers.\\
\\
Let $S = \subseteq \{ \mathbb{N} | n+2$ is the product of 1 or more primes
\begin{align}
  0 \in S && \text{Prove zero is in S}\\
  2 = 2 && \\
  \text{Prove n is in S} && \\
  \text{Suppose that} S \in && \\
  \text{I had to debate that} n = S && \\
  \text{Either} n + 2 \text{is a prime} && \\
  \text{or} && \\
  n + 2 && \text{is composite of 2 or more primes} \\
  (n + 2) & = a \times b & \text{where} (0 < a, b < n+1) \\
  && n \leq b < n \\
  \therefore n \in S &&
\end{align}
18 = 2 * 9
   = 2 * 3 * 3 <-- product of primes
17 = cannot be split because\ldots
   = it is already a prime (1 * 17)

\section{Proofs by Well Ordering}
\label{sec:WellOrdering}
The well ordering principle states that if $S \subseteq \mathbb{N}$ then
either S is empty or S has a smaller element.\\
\\
That is to say, every non-empty subset of the natural numbers has a smallest
element.

\begin{align}
  \sum_{r=0}^{n} 4r^{3} & = n^{2}(n+1)^{2} & \\
%  \text{Let C be the set of counter examples - ie prove that C is empty.} && \nonumber \\
%  \text{Suppose C wasn't empty, then \epsilon has a smallest element, S.} && \nonumber \\
%  S \neq 0 \because & \sum_{r=0}^{0} 4r^{2} = 0 = 0^{2} * (0+1) & \\
%  \text{0 isn't in C, so it cannot be the smallest element of C} &&
\end{align}

\section{Direct Proofs}
\label{sec:DirectProofs}

\section{Proofs by Contradictions}
\label{sec:ProofsByContradictions}