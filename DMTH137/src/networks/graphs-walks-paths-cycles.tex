%-----------------------------------------------------------------------------%
%- Networks :: Walks, Paths and Cycles ---------------------------------------%
%-----------------------------------------------------------------------------%
\chapter{Paths and Circuits}
\label{sec:WalksPathsCircuits}

There are two types of paths and circuits we concern ourselves with, Euler and
Hamilton. An Euler path (or circuit) is one which crosses edges. A Hamilton path
cross vertices.

\section{Eulerian Paths and Circuits}
An Euler \emph{path} is a path in a graph which includes every edge. An Euler
\emph{circuit} is an Euler path which begins and ends at the same vertex.

There are some rules concerning whether a graph has Eulerian Paths and Cycles:
\begin{enumerate}
  \item A connected multigraph with at least $2$ vertices has an Euler circuit
  iff each of its vertices has even degree.
  \item A connected multigraph has an Euler path but \emph{not} an Euler circuit
  iff it has exactly $2$ vertices of odd degree.
\end{enumerate}

Can an Euler path be found on the diagram (fig
\ref{fig:SevenBridgesKonigsberg}) of ``The 7 Bridges of Konigsberg''?

\begin{figure}[!hbt]
\label{fig:SevenBridgesKonigsberg}
\begin{tikzpicture}[node distance   = 4 cm]
  \useasboundingbox (-1,-1) rectangle (11,11); 
  \tikzset{VertexStyle/.style = {shape          = circle,
                                 ball color     = orange,
                                 text           = black,
                                 inner sep      = 2pt,
                                 outer sep      = 0pt,
                                 minimum size   = 24 pt}}
  \tikzset{EdgeStyle/.style   = {thick,
                                 double          = orange,
                                 double distance = 1pt}}
  \tikzset{LabelStyle/.style =   {draw,
                                  fill           = yellow,
                                  text           = red}}
     \node[VertexStyle](A){A};
     \node[VertexStyle,right=of A](B){B};
     \node[VertexStyle,right=of B](C){C};
     \node[VertexStyle,above= 8 cm of B](D){D};     
     \draw[EdgeStyle](B) to node[LabelStyle]{1} (D) ;
     \tikzset{EdgeStyle/.append style = {bend left}}
     \draw[EdgeStyle](A) to node[LabelStyle]{2} (B);
     \draw[EdgeStyle](B) to node[LabelStyle]{3} (A);
     \draw[EdgeStyle](B) to node[LabelStyle]{4} (C);
     \draw[EdgeStyle](C) to node[LabelStyle]{5} (B);
     \draw[EdgeStyle](A) to node[LabelStyle]{6} (D);
     \draw[EdgeStyle](D) to node[LabelStyle]{7} (C);
  \end{tikzpicture}
  \caption{The Seven Bridges of Konigsberg, cf:
  \url{http://www.texample.net/tikz/examples/bridges-of-konigsberg/}}
\end{figure}

\emph{No}. The number of times you enter a circuit equals the number of times
you leave. Each time you enter, you use a different edge. If you enter a vertex
twice, you would have to have used that same entrance twice. Every time you
leave, it is via a different and unused edge.

\section{Hamilton Paths and Circuits}
A Hamilton path passes through every vertex exactly once. A Hamilton circuit
passes through every vertex exactly once and ends at the starting vertex (the
only vertex which can be visited twice).

\section{Minimal Spanning Trees}
A spanning tree in a connected weighted graph with the smallest total weight.

\subsection{Kruskal's Algorithm}
Kruskal's algorithm works by using the smallest weight of each edge each time.

First: Create 3 lists:
\begin{enumerate}
  \item All edges in order of increasing weight
  \item edges in tree (initially empty)
  \item current connected components
\end{enumerate}

Remainder of algorithm runs as follows:
\begin{itemize}
  \item Transfer the lowest weight edge from (1) to (2).
  \item Add the vertices at the edge to the appropriate connected component in (3).
  \item Remove from list (1) any edges which join vertices in the same connected component in (3).
  \item Repeat until list (1) is empty.
  \item List (2) contains your minimal tree.
\end{itemize}

% TODO: include photos

\subsection{Prim's Algorithm}
Prim's algorithm smoothes over the details by being ``not so greedy''.

3 Lists; similar to Kruskal's lists.
\begin{enumerate}
  \item All edges in order of increasing weight.
  \item edges in tree (initially empty).
  \item Vertices included so far (start with any one vertex).
\end{enumerate}

\begin{itemize}
  \item Transfer the lowest weight edge in list (1) to list (2) which has 1 end
  in list (3) and at the other end of this edge to list (3).
  \item Remove any edge from list (1) which join this new vertex in list(3) to any of other vertices in list(3).
  \item Repeat until list(1) empty.
\end{itemize}

% TODO: include photos

\section{Chromatic Polynomial - Colouring Graphs}
\label{sec:ColouringGraphs}
The chromatic polynomial deals with colouring graphs so that adjacent vertices
have different colours (yes, pull out crayons!).

Suppose you wanted to draw up the university timetable. We must ensure that no
student is attending two units at the same time. Consider the graph where the
vertices are the units and two vertices are joined if they have a student in
common.

% TODO scan in diagram
Can I colour the vertices of this graph such that no two adjacent edges have
the same colour?

(Each colour represent a student, and is also indicative of how many
timetable slots are necessary).

Given a graph $G_1$, how many different ways can it be coloured in $K$ colours
so that no two adjacent edges have the same colour?

There is a polynomial that tells us the answer called the \emph{chromatic
polynomial}.

eg (Triangle in TODO). 
$K_3$ using 4 colours.
% Triangle: K=4; use an equilateral triangle

There are 4 different ways to colour the top vertex, 3 different ways to
colour the next, and 2 for the right next after that.
\begin{align}
  4 \times 3 \times 2 & = 24
\end{align}

$K_3$ using 6 colours: 
\begin{align}
  6 \times 5 \times 4 & = 120
\end{align}

$K_3$ using $k$ colours:
\begin{align}
  k \times (k-1) \times (k-2) & = \ldots \\
  & = k^{3} - 3k^{2} + 2k
\end{align}
This is called the \emph{chromatic polynomial for $K_3$}.

% TODO: triangular wheel
$K_4$ using 4 colours:
\begin{align}
  4 \times 3 \times 2 \times 1 & = 24
\end{align}

$K_4$ using $k$ colours:
\begin{align}
  k \times (k-1) \times (k-2) \times (k-3) & = \ldots \\
  & = k^{4} - 6k^{3} + 11k^{2} - 6k
\end{align}
\emph{Chromatic polynomial for $K_4$}

At this point the chromatic polynomial for $n$ vertices with $K_n$ colours:
\begin{align}
  P(k) = k \times (k-1) \times (k-2) \times \ldots \times (k-n+1) & = \ldots
\end{align}

\subsection{Unconnected Vertices in a Coloured Graph}
\label{sec:UnconnectedVerticesInAColouredGraph}
Unconnected vertices can be coloured $K$ times as they are not joined to another
vertex, so we don't worry about what colour we give them.

% 3 vertices no edges (triangle no lines)
3 vertices no edges (triangle no lines)
\begin{align}
  P(k) = k^{3}
\end{align}

4 vertices:
\begin{align}
  P(k) = k^{4}
\end{align}

$n$ vertices:
\begin{align}
  P(k) = k^{n}
\end{align}

\subsection{Trees in Coloured Graphs}
\label{sec:TreesInColouredGraphs}
The chromatic polynomial for any tree on $n$ vertices is
\begin{align}
  k \times (k-1)^{n-1} & = \ldots
\end{align}

Square:

Question: How can I count the number of ways in the case of a square? Just multiplying
won't work because 2 edges will sometimes be
4 3
3 ?

and others

A  B
4  3
2  ?
C  D

Dependent on the different colours chosen.

Answer: We need a clever idea, like, removing an edge:
A  B
+---
|
+---
C  D

This is now a tree, and not the original graph; however, the answer here might help us.
This tree can be coloured in $k(k-1)^{3}$ different ways with $k$ colours.

Colourings where C and D are different are valid colourings for our square. 

For graphs who's C and D have the same colour; will be valid colourings for

% TODO: triangle

These are $k(k-1)(k-2)$.

So
\begin{align}
  P(k) & = k(k-1)^{3} - k(k-1)(k-2) \\
  & = (k-1)(k-1)^{2}k - k((k-2)) \\
  & = \ldots
\end{align}

So for example square:
\begin{align}
  5 \times 4^{3} - 5 \times 4 \times 3
  \intertext{with 5 colours}
  5 \times 64 - 5 \times 12 & = 5 \times \ldots \times 260
\end{align}

\emph{Notion 1:} So far we have the notion that the answer can be determined from a polynomial
we can compute.

\emph{Fact 1:} A graph with $n$ vertices and no edges can be coloured in $k^{n}$
ways.

\emph{Notion 2:} We can reduce the problem of colouring a graph $G$
to a problem involving graphs with fewer edges. \emph{But how?}\\
\\
Consider a graph, $G$ with an edge between vertices A and B (and all sorts of
other edges and vertices). Now consider two graphs formed from $G$, $G_-$ and
$G_=$.\\
\\
$G_-$ is $G$ without the edge AB. \\
$G_=$ is $G$ with edge AB combined into the output (same old edges in it's
output)\\
\\
Notice that a colouring of the graph $G_-$ can either have AB with different
colours, \emph{XOR} AB with the same colour.\\
\\
In first case, we can use the same colouring to colour $G$. In the second case,
edge AB are the same colour, it doesn't matter if A and B are the same colour. \\
\\
This gives rise to \emph{Fact 2:}, the number of ways to colour $G_-$  = 
the number of ways to colour $G$ = the number of ways to colour $G_=$.

\# ways to colour $G$ = \# ways to colour $G_- - $ \# ways to colour $G_=$, or
more mathematically:
\begin{align}
  P_G(k) & = P_{G_-}(k) - P_{G_=}(k)
\end{align}

%G = house
%      A
%    B   C
%    D   E
% now remove ceiling and merge BC into same point:
%
%      A
%      BC
%   D    E
%
% now remove BD
%
%      A
%     BD  C
%         E
%
% now remove CE
%
%      A
%     BD  CE

% Polynomials:
% -> Triangle:
%    $k(k-1)(k-2)$
%    (now factorize)
% -> Tree:
%    $k(k-1)(k-1)(k-1)$
%    (now factorize)