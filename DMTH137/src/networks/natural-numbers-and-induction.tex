%-----------------------------------------------------------------------------%
%- The Natural Numbers and Mathematical Induction ----------------------------%
%-----------------------------------------------------------------------------%
\chapter{Number Theory}
\label{chap:NumberTheory}
Discrete maths concerns itself with 2 main types of numbers:
\begin{enumerate}
  \item Natural numbers, $\mathbb{N}$
  \item Integers, $\mathbb{Z}$
\end{enumerate}

\section{Integers}
\begin{description}
  \item[Notation] The set of integers is denoted by the symbol $\mathbb{Z}$
  \item[Operations] $\mathbb{Z}$ can be added and multiplied.
  \item[Closure Law for Addition]
  \begin{align}
    \forall a, b \in \mathbb{Z}, \exists c \in \mathbb{Z} \\
    a + b = c
  \end{align}
  
  \item[Associative Law for Addition]
  \begin{align}
    a + (b + c) & = (a + b) + c &
  \end{align}
  
  \item[Special case: Zero]
  \begin{align}
    \exists 0 \in \mathbb{Z} \\
    a + 0 & = 0 + a & = a
  \end{align}
  
  \item[Inverse Law for Addition]
  \begin{align}
    \forall a \in \mathbb{Z}, & ~ \exists (-a) \in \mathbb{Z} \\
    a + (-a) & = (-a) + a = 0
  \end{align}
  Theorem: Suppose $b \in \mathbb{Z}$ and $(a+b) = (b+c) = a$ then $b = 0$
  \begin{align}
             b + a & = a \\
      (b+c) + (-a) & = a + (-a) \\
    b + (a + (-a)) & = 0 \\
             b + 0 & = 0 \\
                 b & = a
  \end{align}
  Alternatively:
  \begin{align}
                 b & = b + 0 \\
                   & = b + (a+(-a)) \\
                   & = (b + a) + (-a) \\
                   & = a + -(a) \\
                   & = 0 
  \end{align} \qedbitches
  
  \item[Closure Law for Multiplication]
  \begin{align}
    \forall a, b \in \mathbb{Z} & \Rightarrow \exists c \in \mathbb{Z} \\
    a \times b & = c
  \end{align}
  
  \item[Commutativity Law for Multiplication]
  \begin{align}
    (a \times b) & = (b \times a)
  \end{align}
  
  \item[Associativity Law for Multiplication]
  \begin{align}
    (a \times b) \times c & = a \times (b \times c)
  \end{align}
  
  \item[Identity for Multiplication]
  \begin{align}
    \exists 1 \in \mathbb{Z} & ~ | 1 \neq 0 \\
    \forall a \in \mathbb{Z} & ~ (1 \times a) = a
  \end{align}
  
  \item[Calculation Law for Multiplication]
  \begin{align}
  \text{If }       & a, b, c \in \mathbb{Z} ~ | a \neq 0 \\
                   & (a \times b)     & = (a \times c) \Rightarrow (b = c) & \\
  \text{Then, if } & d \in \mathbb{Z} \text{ \& } && \\
                   & d \times a       & = (a \times d) = a \forall a & \\
  \text{Then }     &&                 & \nonumber \\
                   & d = 1            && \\
  \text{Proof: }   & d = d \times 1   && \\
                   &   = 1            &&
  \end{align}
  \item[Distributive Law for Multiplication]
  \begin{align}
    a(b+c) & = (ab) + (ac)
  \end{align}
\end{description}

\section{Natural Numbers}
\label{sec:NaturalNumbers}
$\mathbb{Z}$ has a special subset called the natural numbers, $\mathbb{N}$
with the following properties:
\begin{description}
  \item[Notation] The set of natural numbers is denoted by the symbol
  $\mathbb{N}$. It imports almost all of the laws of addition and multiplication
  with the following exceptions and properties
  \item[Positive Only Numbers] If $a \in \mathbb{Z}$, exactly one of $(a \times
  -a) \in \mathbb{N}$
  \item[Closure] N is closed under addition or multiplication.\footnote{if you
  add 2 natural numbers, you get a natural number, if you multiply natural numbers,
  you get a natural number}.
  \item[Zero] $0 \in \mathbb{N}$ by definition. ($0 \in \mathbb{Z}$ too)
  \item[Add 1] If $S \subseteq \mathbb{N}$ and $\{0 \in S\}$ then
  \begin{align}
      \{ \text{If } a \in S \} \text{ then } a + 1 & = S \\
      & \Rightarrow S = \mathbb{N}
   \end{align}
\end{description}